\documentclass[11pt]{article}
\usepackage{amsmath}

\title{Essai}
\begin{document}
\maketitle


\tableofcontents
\clearpage

\section{The political consequences of Africa’s mobile revolution - Yeandle, 2023, WP}

\subsection{Abstract}
\begin{itemize}
    \item \textbf{QR}: How mobile technology shapes public opinion in geographically isolated areas ?
    \item \textbf{TH}: Mobile devices increase contact with physically distant social networks
    \\
    $\Rightarrow$ Rising urban contact provides exposure to opposition-aligned political contexts, which reduce government support and political trust
    \item \textbf{ES}: \begin{itemize}
        \item Afrobarometer, Staggered DiD, continent-wide
        \item Household panel, OLS ?, Ghana
    \end{itemize}
    \item \textbf{L}: social contract, accountability, technological change
\end{itemize}

\subsection{Notes}
\subsubsection*{Results}
\begin{itemize}
    \item The arrival of mobile coverage reduces both trust in President and evalutation of government performancen a dynamic effect that continues several years after en area enters reception
    \item Evidence that coverage is not targeted by governments, nor that its future provision has anticipatory effects
    \item The arrival of moblile coverage drives a decline in trust and apparaisal of national but not local politicians
    \begin{itemize}
        \item Rural residents are likely better-informed about local political actors before the arrival of coverage, so it is national figures and insitutions which are subject to change
    \end{itemize}
    \item Negative effects are driven by respondents in the most isolated, distant, rural areas
    \begin{itemize}
        \item Compare individuals living less than 273km away from their capital (the median distance) with those living furhter away than this
    \end{itemize}
    \item Mobile coverage has much smaller informational impacts on urban respondents, and are more exposed to other forms of mass media and direct forms of opposition moblilisation, like protests
\end{itemize}
\subsubsection*{Data}
\begin{itemize}
    \item Coverage is a strong proxy for individual-level, allows to look further back in time, before usage questions were common in social surveys
    \item Afrobarometer rounds 1-6: 1999 to 2015
    \item Respondents are classified as in or outside coverage based on the coordinates of their enumeration area
\end{itemize}
\subsubsection*{Specification}
\begin{itemize}
    \item 1st concern: Expansions are staggered, both within and across countries: locally defined areas receive coverage at different periods in time
    \begin{itemize}
\item Living inside coverage is a dynamic treatment, as it takes time for local residents in newly covered areas to start personnaly using mobile devices, and thus for urban contact and trust effects to ensue
    \\
    $\Rightarrow$: TWFE might draw erroneous comparisons between groups: treatment "turns on" in different periods and has effects that vary over time. This might contaminate the control group under TWFE and lead to "forbidden comparisons" (Goodman-Bacon 2018; Dube et al. 2023)
    \item Run event-study specifications using the Sun and Abraham (2021) estimator: designed for staggered treatments with dynamic effectsn, estimating a series of cohort-by-period interactions which can then be aggregated to produce an overall ATT
\end{itemize}
\end{itemize}
\begin{itemize}
    \item Use never-treated units as the central comparison group (11\% of the full sample), represent en "clean" counterfactual (Dube et al. 2023)
    \item 2nd concern: the outcome data stems from repeated cross-sections surveys of the same population buth with different individuals sampled in each rounds
    \begin{itemize}
    \item As respondents are selected trough randomisation, we can assume the rising probability of living inside coverage in later rounds is driven by the gradual expansion of coverage itself
\end{itemize}
    \item Country and year fixed effects; EA clustering
\end{itemize}

\subsection*{Threats to inference}
\subsubsection*{SUTVA}
\begin{itemize}
    \item "Individuals living inside coverage are affected by it, and those living outside are not"
    \item Spatial nature of the context increases risks of spillovers
    \item 3 solutions:
    \begin{itemize}
        \item Spillovers are likely to bias against finding significant effects, since the negative impact of coverage would be indirectly shaping opinion in control group observations
        \item Spillovers between EAs are unlikely; rural EAs are often physically far from one another due to low population density
        \item Conley SE 50km radius
    \end{itemize}
\end{itemize}
\subsubsection*{Parallel trends}
\begin{itemize}
    \item "If those in the treatment group did not receive coverage, changes in behaviour would not diverge from those under control"
    \item Concerns that treated areas are different in other ways besides coverage, or that coverage might be targeted by governmentrs seeking to change political behaviour
    \item Allowing for country trends help relax this assumption by allowing each country to follow its own path
\end{itemize}
\subsubsection*{No (relevant) anticipation effects}
\begin{itemize}
    \item "Respondents do not condition their behavior on when they will receive coverage in the future"
    \item Seems unlikely: decisions about the future allocation of coverage are taken commercially, in either country capitals or overseas, and it is difficult ti see how rural residenst living outside of coverage would come to know about these
    \item Examining pre-trends, there is no evidence that public opinion shifts in periods just prior to receiving treatment
\end{itemize}
\subsubsection*{Discussion}
\begin{itemize}
    \item Mobile technology appears to exhibit one way negative effects, undercutting previous pro-incumbent bias. It is not clear that rural opinion is reponding to the true quality of politicians, more than it is responding to the homogeneous, and perhaps biased, information they are given by urban relaives
    \item It is an open question whether represents a better or worse state of rural accountability, or whether this shift is normatively desirable
    \item Further work should consider how mobile internet, and related exposure to social media platforms, differs, from basic forms of mobile telephony that this paper investigates
\end{itemize}
\clearpage
\section{Mobile internet and national identity in sub-saharan Africa, Choi, Laughlin, Schultz, 2023, WP}

\subsection{Abstract}
\begin{itemize}
    \item \textbf{QR}: How the expansion of mobile internet infrastructure affects national identity in sub-Saharan Africa ?
    \item \textbf{TH}: Access to mobile internet in diverse society undermines national identity because it facilitates voter exposure to the polarizing tendencies of internet-based social media and communication platforms
    \item \textbf{ES}: \begin{itemize}
        \item DiD; Afrobarometer: internet on identification
        \item As if random variation in the timing of individuals' survey interview relative to presidential elections
    \end{itemize}
    \item \textbf{L}: How technological innovations can inhibit the process of state-building in diverse societies
\end{itemize}

\subsection{Notes}
\subsubsection*{Theory}
\begin{itemize}
    \item Mobile internet can undermine national identification by exposing citizens to internet-based social media platforms which provide two interacting pathways to polarization
    \begin{enumerate}
        \item An elite-driven pathway whereby citizens come into contact with the polarizing rhetoric peddled by political parties and politicians that capitalize on the opportunities provided by these new social media platforms during elections
        \item A mass-driven pathway whereby citizens not only consume or disseminate polarizing mis-disinformation from politicians but also themselves take part in the productuon and reproduction of such mis/disinformation
    \end{enumerate}
\end{itemize}
\subsubsection*{Results}
\begin{itemize}
    \item Expanded access leads to a decrease in individual's propensity to identity with the nation over their ethnocommunal group
    \item Proximity to presidential elections amplifies the polarizing effects of the mobile internet coverage
\\
$\Rightarrow$ Do not intend to adjucate whether polarization through increased mobile access is more an elite-driven or mass-driven process

\end{itemize}
\subsubsection*{Specification}
\begin{itemize}
    \item Afrobarometer: rounds 3-7; 2005-2018; 27 countries
    \item Combining survey responses with mobile coverage spatial polygons allows to locate whether respondents fall within or outside of mobile coverage boundaries
    \item DiD using repeated cross-sections
    \begin{itemize}
    \item 1st difference: whether an individual lives in an area in which, during the period of study, mobile internet coverage becomes available
    \item 2nd difference: whether an individual is surveyed in the pre-mobile coverage or post-mobile coverage period
    \item Compare treated units to "not-yet treated units": period study of 13 years during an era of rapid mobile internet expansion, so individuals who are still left without mobile internet coverage at the end of the period may be fundamentally different
    \end{itemize}

    \item Country/district 
    \begin{itemize}
     \item        \begin{equation*}
        \text{y}_{ijt} = \alpha + \beta \text{Covered}_{ijt} + \gamma \text{EverCovered}_{i} + \delta_j + \zeta_t + \theta \text{X}_{it} + \epsilon_{ijt}
        \end{equation*}
    \item Geographical area (country or district); clustered by locality (robustness with Conley 500km)
\end{itemize}

    \item Grid
    \begin{itemize}
        \item Aggragate the treatment to 55x55km
        \item Round and grid-cell FE
        \item Individuals are treated if the majority of respondents in the grid cell during a round have mobile coverage
        \item Clustered by grid-cell
        \item Staggered problems, not shown in the WP
    \end{itemize}         

\end{itemize}

\subsubsection*{Alternative explanations}
\begin{itemize}
    \item Access to mobile internet may have increased individual repsondents' wealth
    \\
    $\Rightarrow$ Controls measures of wealth
    \item Results driven by the selective expansion of mobile internet coverage into affluent areas
    \\
    $\Rightarrow$ Controls for time-varying measure of population density; distance from a road network
    \item Increased coverage may accompany a broader modernization process
    \\
    $\Rightarrow$ Time varying controls for nighttime light density
    \item Result do not emerge through electoral mechanism but by increased access to information that shapes citizens evaluation (approval or trust) of the government and insitutions
    \\
    $\Rightarrow$ Control for measures of trust; no effect of increased mobile internet on trust
    \item Basic mobile internet specifications
\end{itemize}
\end{document}