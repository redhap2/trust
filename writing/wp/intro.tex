\documentclass[11pt]{article}
\input{Preamble}

\begin{document}

\title{The Death of Distance: Internet and Political Trust in Africa  \\  }

\author{
\textbf{Illan Barriola} \\ CRED -- Paris-Panthéon-Assas \\
\and 
\textbf{R\'{e}dha Chaba} \\ LEMMA -- Paris-Panthéon-Assas  \\
}

\date{\today}

\maketitle

\newpage

\begin{quotation}
\noindent ``The State stops twelve kilometers from the capital.'' 

\smallskip

\hfill -- A Central African Republic official, Bierschenk \& Olivier de Sardan, 1997 
\end{quotation}

\section{Introduction}


Democracy requires citizens to both trust and distrust their politicians. Trust to enable governance, distrust to prevent misgovernance.
Yet, this balance is unevenly distributed across space.
In Sub-Saharan Africa, views of politicians and governments differ by region within countries.
People in remote areas tend to support them more than those near capital cities.
A large literature emphasizes the role of information in shaping political opinion.
Information costs increase with distance from state presence and limited media coverage, resulting in shortages of both direct and mediated information.
When information is scarce, citizens might end up trusting politicians in the absence of negative information.\\

Mobile internet is changing traditional information geography in Sub-Saharan Africa. Citizens in remote areas can now access diverse content that used to be concentrated near capitals - including both political information and misinformation.
This technology differs from traditional media by turning citizens from passive consumers into active participants in different information networks.
The impact on views of politicians is likely to be larger in remote areas, where information costs have been highest.\\

This paper examines whether mobile internet use reduces spatial disparities in political trust across Sub-Saharan Africa.
We combine geocoded individual-level data from Afrobarometer rounds 6-8 (2013-2021) across 17 Sub-Saharan countries with digital maps of mobile internet coverage from the Global System for Mobile Communications Association (GSMA).
The geocoded data allows us to measure each respondent's distance from their capital, while the surveys capture political opinions, internet use, and news consumption.\\

We find that internet use shifts opinions in remote areas toward the critical views found near capitals.
Since we cannot disentangle whether citizens consume accurate information or misinformation, we explore institutional heterogeneity and find that the reduction in spatial disparities in political trust is more pronounced in countries where the state controls traditional media, while the effect on corruption perception is stronger where corruption is high.
This supports "liberation technology" arguments that the internet offers alternative information when traditional media are not transparent.
Moreover, we find that the shift is larger for less educated citizens, consistent with the idea that more educated ones may already have access to information.
Further, we find that the shift in opinion of politicians translates into increased demand for accountability mechanisms and greater willingness to vote against the ruling party in the next elections.
Additionally, we provide evidence of the initial difference in political trust by distance from capital, consistent with the existing literature.\\

We estimate how internet use interacts with distance from capital to affect political trust.
Since internet use is potentially endogenous due to political trust and unobserved factors, we develop two distinct instrumental variable strategies.
First, we use population-weighted 3G internet supply at the district level to instrument for internet use, both interacted with distance from capital.
This follows an intention-to-treat design using 3G supply as a measure of potential internet use, only affecting political trust through this channel.
We argue that 3G supply location is determined by economic rather than political factors, particularly in Sub-Saharan Africa where costly internet infrastructure is deployed by profit-focused private companies.
The population weighting and controls for economic development and topographic features help ensure the exogeneity of our instrument.
Second, we explore another framework with a more exogenous instrument using lightning strike patterns to instrument for 3G internet supply, here also both interacted with distance from capital, population-weighted, and at the district level.
This strategy leverages higher internet infrastructure deployment and maintenance costs in lightning-prone areas, while these weather patterns are plausibly exogenous to political opinions.
We then examine heterogeneous effects by re-estimating our specification across subsamples defined by governance and media freedom measures from V-Dem, World Bank, and Reporters Sans Frontières.
Beyond instrumenting for internet use, we also address potential endogeneity concerns with distance from capital itself by exploiting modern national borders that arbitrarily divide historical ethnic homelands.
Our border discontinuity design compares individuals from the same historical ethnic group who live at different distances from their capital city due to national boundaries.
We estimate a specification including institutional controls to address concerns about comparing political opinions across different national contexts.\\

Hello.


fdf966049
\end{document}