%l !Rnw weave = knitr

\documentclass[11pt]{article}



\usepackage[]{graphicx} % omit 'demo' option in real doc.
\usepackage[]{color}
%% maxwidth is the original width if it is less than linewidth
%% otherwise use linewidth (to make sure the graphics do not exceed the margin)
\usepackage{epstopdf}
\usepackage{float}
\epstopdfDeclareGraphicsRule{.tif}{png}{.png}{convert #1 \OutputFile}
\AppendGraphicsExtensions{.tif}

\makeatletter
\def\maxwidth{ %
  \ifdim\Gin@nat@width\linewidth
    \linewidth
  \else
    \Gin@nat@width
  \fi
}
\makeatother

\definecolor{fgcolor}{rgb}{0.345, 0.345, 0.345}
\newcommand{\hlnum}[1]{\textcolor[rgb]{0.686,0.059,0.569}{#1}}%
\newcommand{\hlstr}[1]{\textcolor[rgb]{0.192,0.494,0.8}{#1}}%
\newcommand{\hlcom}[1]{\textcolor[rgb]{0.678,0.584,0.686}{\textit{#1}}}%
\newcommand{\hlopt}[1]{\textcolor[rgb]{0,0,0}{#1}}%
\newcommand{\hlstd}[1]{\textcolor[rgb]{0.345,0.345,0.345}{#1}}%
\newcommand{\hlkwa}[1]{\textcolor[rgb]{0.161,0.373,0.58}{\textbf{#1}}}%
\newcommand{\hlkwb}[1]{\textcolor[rgb]{0.69,0.353,0.396}{#1}}%
\newcommand{\hlkwc}[1]{\textcolor[rgb]{0.333,0.667,0.333}{#1}}%
\newcommand{\hlkwd}[1]{\textcolor[rgb]{0.737,0.353,0.396}{\textbf{#1}}}%

\usepackage{framed}
\makeatletter
\newenvironment{kframe}{%
 \def\at@end@of@kframe{}%
 \ifinner\ifhmode%
  \def\at@end@of@kframe{\end{minipage}}%
  \begin{minipage}{\columnwidth}%
 \fi\fi%
 \def\FrameCommand##1{\hskip\@totalleftmargin \hskip-\fboxsep
 \colorbox{shadecolor}{##1}\hskip-\fboxsep
     % There is no \\@totalrightmargin, so:
     \hskip-\linewidth \hskip-\@totalleftmargin \hskip\columnwidth}%
 \MakeFramed {\advance\hsize-\width
   \@totalleftmargin\z@ \linewidth\hsize
   \@setminipage}}%
 {\par\unskip\endMakeFramed%
 \at@end@of@kframe}
\makeatother

\definecolor{shadecolor}{rgb}{.97, .97, .97}
\definecolor{messagecolor}{rgb}{0, 0, 0}
\definecolor{warningcolor}{rgb}{1, 0, 1}
\definecolor{errorcolor}{rgb}{1, 0, 0}
\newenvironment{knitrout}{}{} % an empty environment to be redefined in TeX

\usepackage{alltt}

\usepackage{hyperref}
\hypersetup{
        colorlinks=true,
        breaklinks,
        allcolors=[RGB]{128,0,0}
}

\let\oldFootnote\footnote
\newcommand\nextToken\relax

\renewcommand\footnote[1]{%
    \oldFootnote{#1}\futurelet\nextToken\isFootnote}

\newcommand\isFootnote{%
    \ifx\footnote\nextToken\textsuperscript{,}\fi}

\usepackage[utf8]{inputenc}
\usepackage[T1]{fontenc}
\usepackage{crimson}

\usepackage{geometry}
\geometry{verbose,tmargin=1in,bmargin=1in,lmargin=1in,rmargin=1in}
\usepackage{url}
\usepackage{dcolumn}
\usepackage{ctable}
\usepackage{booktabs}
\usepackage{multirow}
\usepackage{setspace}
\usepackage{rotating}
\usepackage{graphicx}
\usepackage{subcaption}
\usepackage{seqsplit}
\usepackage{amsmath,amsfonts,amssymb,amsthm}
\usepackage{soul}
\usepackage{float}
%\usepackage[multiple]{footmisc}
%\usepackage{longtable}


\usepackage{bbding} % checkmark symbol
\usepackage{comment} % checkmark symbol

\renewcommand{\textfraction}{0.05}
\renewcommand{\topfraction}{0.8}
\renewcommand{\bottomfraction}{0.8}
\renewcommand{\floatpagefraction}{0.75}

%Bibliography
\usepackage{natbib}
%\usepackage[autostyle]{csquotes}
%\usepackage[backend=bibtex, style=authoryear, natbib=true]{biblatex}
%\addbibresource{CCH.bib}

%%For Hypotheses and Subhypotheses
%\usepackage{ntheorem}
\newtheorem{hyp}{Hypothesis}
\makeatletter
\newcounter{subhyp}
\let\savedc@hyp\c@hyp
\newenvironment{subhyp}
 {%
  \setcounter{subhyp}{0}%
  \stepcounter{hyp}%
  \edef\saved@hyp{\thehyp}% Save the current value of hyp
  \let\c@hyp\c@subhyp     % Now hyp is subhyp
  \renewcommand{\thehyp}{\saved@hyp\alph{hyp}}%
 }
 {}
\newcommand{\normhyp}{%
  \let\c@hyp\savedc@hyp % revert to the old one
  \renewcommand\thehyp{\arabic{hyp}}%
}
\makeatother
%%%

\setcounter{MaxMatrixCols}{10}

\theoremstyle{plain}
\newtheorem{prop}{\protect\propositionname}
\theoremstyle{plain}
\newtheorem{thm}{\protect\theoremname}
\makeatother
\providecommand{\examplename}{Example}
\providecommand{\propositionname}{Proposition}
\providecommand{\theoremname}{Theorem}
\newcommand{\bi}{\begin{itemize}}
\newcommand{\ei}{\end{itemize}}
\newcommand{\bb}{\begin{block}}
\newcommand{\eb}{\end{block}}
\newcommand{\bmath}{\begin{eqnarray}}
\newcommand{\emath}{\end{eqnarray}}
\newcommand{\bmathnn}{\begin{eqnarray*}}
\newcommand{\emathnn}{\end{eqnarray*}}
\newtheorem{theorem}{{Theorem}}%[section]
\newtheorem{proposition}[theorem]{Proposition}
\newtheorem{lemma}[theorem]{Lemma}
\newtheorem{definition}[theorem]{Definition}
\newtheorem{prediction}{Prediction}
\newtheorem{open_question}{Open question}
\newtheorem{assumption}[theorem]{Assumption}
\newtheorem{observation}[theorem]{Observation}
\newtheorem{claim}[theorem]{{Claim}}
\newtheorem{example}[theorem]{{Example}}
\newtheorem{corollary}[theorem]{{Corollary}}
\newtheorem{remark}[theorem]{{Remark}}
\newtheorem{assumptions}[theorem]{{Assumptions}}
\newtheorem*{thm1star}{{Theorem $\mathbf{1^*}$}}
\newtheorem*{thm2star}{{Theorem $\mathbf{2^*}$}}
\newtheorem*{theorem*}{Theorem}
\newtheorem*{lemma*}{Lemma}


%For nice tables
\usepackage{booktabs}
\usepackage{array}
\usepackage{threeparttable}
\usepackage{tabulary}
\newcolumntype{M}[1]{{\centering\arraybackslash}m{#1}}

%Expectation opperator
\newcommand{\Expect}{{\rm I\kern-.3em E}}

%Significance commands
\newcommand*{\SuperScriptSameStyle}[1]{%
  \ensuremath{%
    \mathchoice
      {{}^{\displaystyle #1}}%
      {{}^{\textstyle #1}}%
      {{}^{\scriptstyle #1}}%
      {{}^{\scriptscriptstyle #1}}%
  }%
}

\newcommand*{\oneS}{\SuperScriptSameStyle{*}}
\newcommand*{\twoS}{\SuperScriptSameStyle{**}}
\newcommand*{\threeS}{\SuperScriptSameStyle{*{*}*}}
\newcommand{\vh}[1]{\textcolor{red}{(VH: #1)}}
\IfFileExists{upquote.sty}{\usepackage{upquote}}{}

\begin{document}


\title{The Death of Distance: Mobile Internet
and Political Trust in Africa\\  \vspace{0.75cm} \underline{Preliminary and incomplete }}

\vspace{1cm}

\author{
\textbf{Illan Barriola} \\ CRED -- Paris II \\
\and \textbf{R\'{e}dha Chaba}\\ LEMMA -- Paris II \\
}

\date{ \today}

\maketitle
\section*{Data}
Figure 1 displays our sample of 20 Sub-Saharan African countries (in green) and their capital cities (in red). Our analysis combines Afrobarometer survey rounds 5-8 (2011-2021), providing geocoded data on public opinion, information access, and demographic characteristics for 98,235 individual respondents across these countries.\footnote{Our sample includes: Benin, Burkina Faso, Botswana, Cameroon, Ivory Coast, 
Ghana, Guinea, Kenya, Liberia, Mali, Malawi, Mozambique, Namibia, Niger, Nigeria, Sierra Leone, Tanzania, Uganda, Zambia, Zimbabwe. We exclude islands, countries with multiple capitals, and Northern African countries. We also exclude countries with missing data on mobile internet coverage and those not present in all survey rounds.}\footnote{All Afrobarometer surveys were conducted via-in-person, face-to-face interviews. The method did not change to computer-assisted interviews, thereby avoiding potential bias toward areas with mobile internet coverage.}

To measure political trust and accountability, we focus on three sets of variables from the Afrobarometer surveys. First, we construct our main dependent variable \textit{Political Trust} by averaging responses to trust questions about the president and parliament/national assembly, each measured on a 0-3 scale ('Not at all' to 'A lot').\footnote{The specific questions ask: \textit{``How much do you trust each of the following: the president, the parliament/national assembly?''}} Second, we measure electoral accountability through a binary variable indicating willingness to vote against the ruling party in hypothetical next-day elections.\footnote{Based on responses to: \textit{``If [presidential elections] were held tomorrow, which [party's candidate] would you vote for?''}} Third, we capture economic performance assessment through a 1-5 scale rating of economic conditions compared to twelve months prior.\footnote{Based on responses to: \textit{``Looking back, how do you rate economic conditions in this country compared to twelve months ago?''}}


To measure remoteness, we calculate each respondent's distance from their national capital using Afrobarometer's geocoded enumeration areas. Following Michalopoulos and Papaioannou (2014), we standardize distances by dividing each respondent's distance from the capital by the maximum distance within their country. This relative distance measure ranges from 0 (at capital) to 1 (furthest point), enabling meaningful cross-country comparisons.


We combine our Afrobarometer data with GSMA's digital maps of mobile internet coverage, providing 1x1-kilometer grid cell resolution data from mobile network providers. For regional coverage measurement, following Guriev et al. (2021) we overlay mobile coverage maps with population density data and calculate weighted averages of internet availability by region and year, using normalized population density weights. This approach ensures comparable coverage measures across regions of different sizes and population distributions.\\
The second part of our empirical strategy addresses how mobile internet expansion affects spatial disparities in political trust. To address potential omitted variable bias - arising from non-random internet infrastructure placement and the correlation between economic development and internet access - we instrument internet coverage using regional lightning strike patterns, following recent literature (Manacorda and Tesei, 2020; Guriev et al., 2021; Cariolle and Carolle, 2024). This identification strategy exploits the fact that areas with frequent lightning strikes face higher infrastructure deployment and maintenance costs, while these weather patterns are plausibly exogenous to political trust. Following Guriev et al. (2021) and Cariolle and Carolle (2024), we construct our instrument by calculating the average daily lightning strikes at the regional level using VHRFC data over 1998-2013, weighted by regional population density in 2011. This weighting accounts for two offsetting forces: while lightning strikes increase infrastructure costs, population density reduces per-capita costs by spreading infrastructure investments across more users in high-risk areas. This instrument provides plausibly exogenous variation in internet access across varying distance from the capital city.

\end{document}