%l !Rnw weave = knitr

\documentclass[11pt]{article}



\usepackage[]{graphicx} % omit 'demo' option in real doc.
\usepackage[]{color}
%% maxwidth is the original width if it is less than linewidth
%% otherwise use linewidth (to make sure the graphics do not exceed the margin)
\usepackage{epstopdf}
\usepackage{float}
\epstopdfDeclareGraphicsRule{.tif}{png}{.png}{convert #1 \OutputFile}
\AppendGraphicsExtensions{.tif}

\makeatletter
\def\maxwidth{ %
  \ifdim\Gin@nat@width\linewidth
    \linewidth
  \else
    \Gin@nat@width
  \fi
}
\makeatother

\definecolor{fgcolor}{rgb}{0.345, 0.345, 0.345}
\newcommand{\hlnum}[1]{\textcolor[rgb]{0.686,0.059,0.569}{#1}}%
\newcommand{\hlstr}[1]{\textcolor[rgb]{0.192,0.494,0.8}{#1}}%
\newcommand{\hlcom}[1]{\textcolor[rgb]{0.678,0.584,0.686}{\textit{#1}}}%
\newcommand{\hlopt}[1]{\textcolor[rgb]{0,0,0}{#1}}%
\newcommand{\hlstd}[1]{\textcolor[rgb]{0.345,0.345,0.345}{#1}}%
\newcommand{\hlkwa}[1]{\textcolor[rgb]{0.161,0.373,0.58}{\textbf{#1}}}%
\newcommand{\hlkwb}[1]{\textcolor[rgb]{0.69,0.353,0.396}{#1}}%
\newcommand{\hlkwc}[1]{\textcolor[rgb]{0.333,0.667,0.333}{#1}}%
\newcommand{\hlkwd}[1]{\textcolor[rgb]{0.737,0.353,0.396}{\textbf{#1}}}%

\usepackage{framed}
\makeatletter
\newenvironment{kframe}{%
 \def\at@end@of@kframe{}%
 \ifinner\ifhmode%
  \def\at@end@of@kframe{\end{minipage}}%
  \begin{minipage}{\columnwidth}%
 \fi\fi%
 \def\FrameCommand##1{\hskip\@totalleftmargin \hskip-\fboxsep
 \colorbox{shadecolor}{##1}\hskip-\fboxsep
     % There is no \\@totalrightmargin, so:
     \hskip-\linewidth \hskip-\@totalleftmargin \hskip\columnwidth}%
 \MakeFramed {\advance\hsize-\width
   \@totalleftmargin\z@ \linewidth\hsize
   \@setminipage}}%
 {\par\unskip\endMakeFramed%
 \at@end@of@kframe}
\makeatother

\definecolor{shadecolor}{rgb}{.97, .97, .97}
\definecolor{messagecolor}{rgb}{0, 0, 0}
\definecolor{warningcolor}{rgb}{1, 0, 1}
\definecolor{errorcolor}{rgb}{1, 0, 0}
\newenvironment{knitrout}{}{} % an empty environment to be redefined in TeX

\usepackage{alltt}

\usepackage{hyperref}
\hypersetup{
        colorlinks=true,
        breaklinks,
        allcolors=[RGB]{128,0,0}
}

\let\oldFootnote\footnote
\newcommand\nextToken\relax

\renewcommand\footnote[1]{%
    \oldFootnote{#1}\futurelet\nextToken\isFootnote}

\newcommand\isFootnote{%
    \ifx\footnote\nextToken\textsuperscript{,}\fi}

\usepackage[utf8]{inputenc}
\usepackage[T1]{fontenc}
\usepackage{crimson}

\usepackage{geometry}
\geometry{verbose,tmargin=1in,bmargin=1in,lmargin=1in,rmargin=1in}
\usepackage{url}
\usepackage{dcolumn}
\usepackage{ctable}
\usepackage{booktabs}
\usepackage{multirow}
\usepackage{setspace}
\usepackage{rotating}
\usepackage{graphicx}
\usepackage{subcaption}
\usepackage{seqsplit}
\usepackage{amsmath,amsfonts,amssymb,amsthm}
\usepackage{soul}
\usepackage{float}
%\usepackage[multiple]{footmisc}
%\usepackage{longtable}


\usepackage{bbding} % checkmark symbol
\usepackage{comment} % checkmark symbol

\renewcommand{\textfraction}{0.05}
\renewcommand{\topfraction}{0.8}
\renewcommand{\bottomfraction}{0.8}
\renewcommand{\floatpagefraction}{0.75}

%Bibliography
\usepackage{natbib}
%\usepackage[autostyle]{csquotes}
%\usepackage[backend=bibtex, style=authoryear, natbib=true]{biblatex}
%\addbibresource{CCH.bib}

%%For Hypotheses and Subhypotheses
%\usepackage{ntheorem}
\newtheorem{hyp}{Hypothesis}
\makeatletter
\newcounter{subhyp}
\let\savedc@hyp\c@hyp
\newenvironment{subhyp}
 {%
  \setcounter{subhyp}{0}%
  \stepcounter{hyp}%
  \edef\saved@hyp{\thehyp}% Save the current value of hyp
  \let\c@hyp\c@subhyp     % Now hyp is subhyp
  \renewcommand{\thehyp}{\saved@hyp\alph{hyp}}%
 }
 {}
\newcommand{\normhyp}{%
  \let\c@hyp\savedc@hyp % revert to the old one
  \renewcommand\thehyp{\arabic{hyp}}%
}
\makeatother
%%%

\setcounter{MaxMatrixCols}{10}

\theoremstyle{plain}
\newtheorem{prop}{\protect\propositionname}
\theoremstyle{plain}
\newtheorem{thm}{\protect\theoremname}
\makeatother
\providecommand{\examplename}{Example}
\providecommand{\propositionname}{Proposition}
\providecommand{\theoremname}{Theorem}
\newcommand{\bi}{\begin{itemize}}
\newcommand{\ei}{\end{itemize}}
\newcommand{\bb}{\begin{block}}
\newcommand{\eb}{\end{block}}
\newcommand{\bmath}{\begin{eqnarray}}
\newcommand{\emath}{\end{eqnarray}}
\newcommand{\bmathnn}{\begin{eqnarray*}}
\newcommand{\emathnn}{\end{eqnarray*}}
\newtheorem{theorem}{{Theorem}}%[section]
\newtheorem{proposition}[theorem]{Proposition}
\newtheorem{lemma}[theorem]{Lemma}
\newtheorem{definition}[theorem]{Definition}
\newtheorem{prediction}{Prediction}
\newtheorem{open_question}{Open question}
\newtheorem{assumption}[theorem]{Assumption}
\newtheorem{observation}[theorem]{Observation}
\newtheorem{claim}[theorem]{{Claim}}
\newtheorem{example}[theorem]{{Example}}
\newtheorem{corollary}[theorem]{{Corollary}}
\newtheorem{remark}[theorem]{{Remark}}
\newtheorem{assumptions}[theorem]{{Assumptions}}
\newtheorem*{thm1star}{{Theorem $\mathbf{1^*}$}}
\newtheorem*{thm2star}{{Theorem $\mathbf{2^*}$}}
\newtheorem*{theorem*}{Theorem}
\newtheorem*{lemma*}{Lemma}


%For nice tables
\usepackage{booktabs}
\usepackage{array}
\usepackage{threeparttable}
\usepackage{tabulary}
\newcolumntype{M}[1]{{\centering\arraybackslash}m{#1}}

%Expectation opperator
\newcommand{\Expect}{{\rm I\kern-.3em E}}

%Significance commands
\newcommand*{\SuperScriptSameStyle}[1]{%
  \ensuremath{%
    \mathchoice
      {{}^{\displaystyle #1}}%
      {{}^{\textstyle #1}}%
      {{}^{\scriptstyle #1}}%
      {{}^{\scriptscriptstyle #1}}%
  }%
}

\newcommand*{\oneS}{\SuperScriptSameStyle{*}}
\newcommand*{\twoS}{\SuperScriptSameStyle{**}}
\newcommand*{\threeS}{\SuperScriptSameStyle{*{*}*}}
\newcommand{\vh}[1]{\textcolor{red}{(VH: #1)}}
\IfFileExists{upquote.sty}{\usepackage{upquote}}{}

\begin{document}


\title{Mobile Internet and Nation Building in Africa\\  \vspace{0.75cm} \underline{Preliminary and incomplete }}

\vspace{1cm}

\author{
\textbf{Illan Barriola} \\ CRED -- Paris II \\
\and \textbf{R\'{e}dha Chaba}\\ LEMMA -- Paris II \\
% \textbf{Paul Maarek}\thanks{\emph{Corresponding author}. LEMMA, Universit\'{e} Panth\'{e}on-Assas (Paris II), France; \texttt{Paul.Maarek@u-paris2.fr}.}   \\ LEMMA -- Paris II  \\
%   \and \textbf{Victor Hiller} \\ LEMMA -- Paris II \\ 
% \and \textbf{Michael T. Dorsch}
%  \\ CEU -- Vienna 
% \and
% \and \textbf{R\'{e}dha Chaba} \\ LEMMA -- Paris II  \\
}

\date{ \today}

\maketitle

\section*{Research question}

How does the diffusion of mobile internet affect political accountability in remote areas ?
\section*{Hypothesis}
\begin{enumerate}
  \item Living in remote areas is associated with higher levels of institutional trust.
  \item Increased internet access mitigates the spatial disparities in the levels of institutional trust.
\end{enumerate}

\section*{Presentation}
\textbf{Many African capital cities are located close to the coast} 
  \begin{itemize}
    \item Constrain the penetration of institutions in the hinterland\\
    $\Rightarrow$ Coexistence of a dual institutional framework within countries (Lewis 1954; Migdal 1988, Michalopoulos and Papaioannou 2014)\\
    $\Rightarrow$ Concentration of the economic activity around capital cities (Pinkovskiy 2013, Provenzano 2024)
  
  \end{itemize}
\clearpage

\textbf{State of Art}\\

The identity of sovereign states and their perceived ligitimacy have dimensions both objective and subjective.
Discrepancies between these two dimensions may appear during processes of democratization because the participatory aspirations of citizens
 and their respective identites will not necessarily coincide with the existing political boundaries and the institutional framework.\\
 
This may lead to a more or less peaceful redrawing of boundaries ands attempts of internal democratic reforms.
These processes and possible conflicts cannot be resolved by democratic standards and procedures because the rule of law pre-supposes an existing poltical unit, and procedures such as majority decisions may exclude and possibly suppress important segments of the population.
If democracy means "rule of the people", it first has to be decided who the people are and which boundaries should be respected.
In this sens, state formation and nation-building must be considered as prerequisites of any meaningful democratization.\\


\textbf{Sow, 2023}\\

The post-colonial era in Africa saw a wave of nation building initiatives aiming to construct a sens of national identity.
The rational for nation building stems from the importance of having a united population for the sake of conflict prevention and long-term political stability.
On key component of nation building is constructing a sense of belonging to one nation and having a strong sense of national identity.
In African countries where ethnic identities remain salient, ethnic ties can be formed at the detriment of national identity. In contexts where ethnic identities are strong and present, individual may feel closer to their ethnic group than their nation, as a whole.\\

The instrumentalist theory defines identity as a malleable social construct, which changes due to a myriad of factors (\textcolor{blue}{\textbf{Williams, 2015}}).\\

Improvements in ICT, leads to an improved access to information.
This improved access to information could heighten the perception of corruption.
Moroever, improved ICT leads to the strenghterning of social networks along digital lines, thus bringing co-ethnics together in the digital world.\\

Nation building is defined as the process of creating and maintaining a shared national identity.\\
Despite the existence of differences - stemming from a different histrorical background, ethnic diversity, a colonial divide and conquer legacy, etc
- among groups, nation building strives to build collective capacity and create a shared vision of the future (\textcolor{blue}{\textbf{Bourgon, 2010}}).\\
Many identify with both their nation and their ethnic group. The position of one's ethnic group, in society, can determine the level of national identity.
\textcolor{blue}{\textbf{Green (2020)}} argues that certain nations have "core" ethnic groups and when the core group is in power, members of the group tend to identify more with the nation.\\

The modernist view states that national identity is a product of modernity, through mechanisms related to industrialization and mass education (\textcolor{blue}{\textbf{Storm, 2018}}).
As countries develop, residents starts to trade in other forms of identity for a more solid national identity. Richer areas tend to identify more with the nation - over their ethnic group - 
due to a lower reliance on ethnic networks (\textcolor{blue}{\textbf{Ahlerup, Baskaran and Bigsten, 2017}}).\\

Centralized political systems encourage citizens to pool their demands into one that suits the wider group of citizens, and abandon demands rooted in provincial origins (\textcolor{blue}{\textbf{Acemoglu, Robinson and Torvik, 2020}}).\\

In Africa notably, geographic features have affected state-building \textcolor{blue}{\textbf{(Herbst, 2000})}. Countries use spatial tools to create or improve national identity.\\
Placement of strategic cities within the nation. Some leaders have shifted the location of their capital in order to make governments more central and closer to their citizens. In 1976, Nigeria moved the capital from Lagos to Abuja.\\

Ethnic identification refers to "a person's use of racial, national or religious terms to identify himself, and thereby, to relate himself to others." (\textcolor{blue}{\textbf{Glaser, 1958}}).
Instrumentalism defines ethnicity as a tool used by individuals of the group to unify and mobilize each other in the aim to achieve a common goal.
Ethnic identities are socially constructed and can be formed through differents channels (\textcolor{blue}{\textbf{Wimmer, 2008}}), such as economic, political, security, and nation building.\\

Research papers using Afrobarometer surveys in Uganda (\textcolor{blue}{\textbf{Green, 2020}}) and Benin (\textcolor{blue}{\textbf{Koter, 2019}}) find that ethnic identification is affected by whether
 or not the president is part of a respondent's ethnic group.
 Perceived exclusion of an ethnic group from power strengthens ethnic identification. Politicians engage in ethnically targeted outreach, which can lead to stronger ethnic identification in both the in-groups who see their ethnic groups catered to and the out-group
  who can perceive a threat through the exclusion (\textcolor{blue}{\textbf{Higashijima and Nakai, 2015}}).\\

Suspect corruption to play a role in poor confidence in government and the subsequent decline in national identification. News consumption and internet access. Also, 
¨possible stronger ethnic connections in the digital world.\\

Nation building is one key pillar of a country's political stability. A lack of thereof can hinder long term economic development by preventing the influx of investment.\\
One key element of nation building is creating a nation of people who feel strong ties to their countries and identify with their nation to some extent. Africa is home to nearly 3000 ethnic groups.
The salience of ethnic identities can take away from national identity and hinder long term nation builiding efforts.\\


\textbf{McKay, 2023}

How trust is spatially distributed matters for how polities are governed, and is consequential for electoral behaviour, trends and outcomes.\\


\textbf{Dellmuth, 2023}

Future research could usefully theorise and examine how the effect of regional inequalities on political attitudes might be conditioned by democracy.\\

\textbf{States in the Developing World, Centeno, Kohli, Yashar, Mistree}\\

- \textbf{Unpacking States in the Developing World: Capacity, Performance, and Politics - Centeno, Kohli, Yashar}

In its simplest terms, state capacity involves the bureaucratic, managerial, and organizational ability to process information, implement policies, and maintain governing systems. State capacity is thus a function of the organizational skills and institutions required for carrying out relevant tasks. In particular, we identify organizational capacity in terms of the following factors:\\
- Presence of the state: To what degree do state penetrates their societies? Mann(1993) highlighted the importance of the infrastructural reach of the state. Mann's critical insight came from recognizing that infrastructural power comes from increasing the level and quality of contact citizens have with state. Organizational capacity is partly conditioned by productive interactions that take place throughout a country's territory.\\

States are not simply about order and growth; they fundamentally depend on the construction, maintenance, and allegiance (sometimes coerced) of a political community (voir les ref, p.20 du livre). States must have mechanisms for structuring and ingraining social inclusion. By social inclusion we mean the ability of the state to incorporate the entire population, to promote social wellbeing, and to establish itself as the property of no particular group or sector.\\

Even where polities have clearly and expansively defined the rules of membership, the question is whether the state can uphold those terms. Can they identify, count, and regulate the citizens of their country. It entails meaningful state capacity throughout a territory and across social cleavages to administer censuses, provide passports, construct participatory institutions, and externd social services. In this regard, social inclusion requires defining citizenship membership and boundaries alongside policies that seek to equalize further political, social, and/or economic rights (Marshall 1963). As should be clear, social inclusion can be used for both democratic and authoritatian ends.\\

-\textbf{State Capacity, History, Structure, and Political Contestation in Africa - Mkandawire}

Concern over state capacity and its deployment in Africa has changed over the years. In the immediate postcolonial era, the concerns revolved around three issues. [...]\\
The second was about "nation building" including especially the capacity to hold together the new nation. This required capcities to inculcate new values of nationalism in the populace, to legitimize the poltical and economic order (Deutsch 1978), and to exercise "infrastructural power" (Mann 1984) that would obviate or diminish the constant recourse to repression.\\

A widespread view was that in the special conditions of Africa, the task of acquiring legitimacy was inherently more complex than in other societies - what with the artificiality if national borders, the multiplicity of ethnic groups, and a populist streak in African politics -  a populism has something to do with the peculiarities of the African agrarian structures and the varied social origins of the "founders" of the new nation.\\

Improve the welfare of the citizenry and thereby enhance its own legitimacy. Similarly, legitimacy can facilitate the extractive capacity of the state through the "quasi-voluntary compliance" it induces among taxpayers (Levi 1988).\\

\textbf{Institutions and Democracy in Africa - Nic Cheeseman}\\
- \textbf{Decentralisation - Accountability in Local Government - Alex Dyzenhaus}\\

African presidents have historically controlled politics on the continent through highly centralised states, a situation that is enrishned in around 80 per cent of African constitutions (Kuperman 2015).\\
The existing literature highlights a number of factors that shape the quality of accountability in devolved systems, such as the strengh of citizens and civil society (Sisk 2001).\\

\textbf{Politics in the Developing World - Burnell, Rakner, Randall}\\
- \textbf{Ethnopolitics and Nationalism - Scarritt, Birnir}\\

National identities are inherntly political, emphasizing the autonomy and unity of the nation as an actual or potential political unit (Hutchinson and Smith 1994). Civic national identities involve unity among citizens of an autonomous state. [...] Very few natioanl identities in the developing world are purely civic, but a substantial majority of them contain civic or multi-ethnic aspects, so that they do not identify the nation with a single ethnic group. Since nationalism is a constructed identity, the significant variations in the specific nature of nations in the developing world are not surprising. The boundaries of the vast majority of developing states were determined by colonial rulers, and the varieties of nationalism are products of the interaction between the states that rule within these boundaries and the morphology of ethnopolitical identities, the tactics of ethnopolitical groups, and the presence of alternative identities within the same boundaries. "The normative model of the contemporary polity calls for the coincidence of nation and state" (Young 1976).\\

Sub-Saharan Africa is where multi-ethnic nationalism is most common, although ethno-nationalism is by no means absent there. The predominence of ethnopolitical cleavages, their complex multilevel morphology, the absence of large cultural differences in most African countries except those divided by religion, and the politics of communal contention combine to produce multi-ethnic national identities  the most effectively integrate national and ethnopolitical identities in this context.\\

National identities: Inherently political, emphasizing the autonomy and unity of nation as an actual or potential political unit.\\
Nation-building: Referring to building a sense of national belonging and unity.\\

\textbf{Trust but verify? Examining the role of trust in institutions in the spread of unverified information on social media - Zoonen et al. 2024}\\

Trust in institutions is associated with reasons for sharing unverified information and reduces individual's motivation to authenticate information.\\


\end{document}