%l !Rnw weave = knitr

\documentclass[11pt]{article}



\usepackage[]{graphicx} % omit 'demo' option in real doc.
\usepackage[]{color}
%% maxwidth is the original width if it is less than linewidth
%% otherwise use linewidth (to make sure the graphics do not exceed the margin)
\usepackage{epstopdf}
\usepackage{float}
\epstopdfDeclareGraphicsRule{.tif}{png}{.png}{convert #1 \OutputFile}
\AppendGraphicsExtensions{.tif}

\makeatletter
\def\maxwidth{ %
  \ifdim\Gin@nat@width\linewidth
    \linewidth
  \else
    \Gin@nat@width
  \fi
}
\makeatother

\definecolor{fgcolor}{rgb}{0.345, 0.345, 0.345}
\newcommand{\hlnum}[1]{\textcolor[rgb]{0.686,0.059,0.569}{#1}}%
\newcommand{\hlstr}[1]{\textcolor[rgb]{0.192,0.494,0.8}{#1}}%
\newcommand{\hlcom}[1]{\textcolor[rgb]{0.678,0.584,0.686}{\textit{#1}}}%
\newcommand{\hlopt}[1]{\textcolor[rgb]{0,0,0}{#1}}%
\newcommand{\hlstd}[1]{\textcolor[rgb]{0.345,0.345,0.345}{#1}}%
\newcommand{\hlkwa}[1]{\textcolor[rgb]{0.161,0.373,0.58}{\textbf{#1}}}%
\newcommand{\hlkwb}[1]{\textcolor[rgb]{0.69,0.353,0.396}{#1}}%
\newcommand{\hlkwc}[1]{\textcolor[rgb]{0.333,0.667,0.333}{#1}}%
\newcommand{\hlkwd}[1]{\textcolor[rgb]{0.737,0.353,0.396}{\textbf{#1}}}%

\usepackage{framed}
\makeatletter
\newenvironment{kframe}{%
 \def\at@end@of@kframe{}%
 \ifinner\ifhmode%
  \def\at@end@of@kframe{\end{minipage}}%
  \begin{minipage}{\columnwidth}%
 \fi\fi%
 \def\FrameCommand##1{\hskip\@totalleftmargin \hskip-\fboxsep
 \colorbox{shadecolor}{##1}\hskip-\fboxsep
     % There is no \\@totalrightmargin, so:
     \hskip-\linewidth \hskip-\@totalleftmargin \hskip\columnwidth}%
 \MakeFramed {\advance\hsize-\width
   \@totalleftmargin\z@ \linewidth\hsize
   \@setminipage}}%
 {\par\unskip\endMakeFramed%
 \at@end@of@kframe}
\makeatother

\definecolor{shadecolor}{rgb}{.97, .97, .97}
\definecolor{messagecolor}{rgb}{0, 0, 0}
\definecolor{warningcolor}{rgb}{1, 0, 1}
\definecolor{errorcolor}{rgb}{1, 0, 0}
\newenvironment{knitrout}{}{} % an empty environment to be redefined in TeX

\usepackage{alltt}

\usepackage{hyperref}
\hypersetup{
        colorlinks=true,
        breaklinks,
        allcolors=[RGB]{128,0,0}
}

\let\oldFootnote\footnote
\newcommand\nextToken\relax

\renewcommand\footnote[1]{%
    \oldFootnote{#1}\futurelet\nextToken\isFootnote}

\newcommand\isFootnote{%
    \ifx\footnote\nextToken\textsuperscript{,}\fi}

\usepackage[utf8]{inputenc}
\usepackage[T1]{fontenc}
\usepackage{crimson}

\usepackage{geometry}
\geometry{verbose,tmargin=1in,bmargin=1in,lmargin=1in,rmargin=1in}
\usepackage{url}
\usepackage{dcolumn}
\usepackage{ctable}
\usepackage{booktabs}
\usepackage{multirow}
\usepackage{setspace}
\usepackage{rotating}
\usepackage{graphicx}
\usepackage{subcaption}
\usepackage{seqsplit}
\usepackage{amsmath,amsfonts,amssymb,amsthm}
\usepackage{soul}
\usepackage{float}
%\usepackage[multiple]{footmisc}
%\usepackage{longtable}


\usepackage{bbding} % checkmark symbol
\usepackage{comment} % checkmark symbol

\renewcommand{\textfraction}{0.05}
\renewcommand{\topfraction}{0.8}
\renewcommand{\bottomfraction}{0.8}
\renewcommand{\floatpagefraction}{0.75}

%Bibliography
\usepackage{natbib}
%\usepackage[autostyle]{csquotes}
%\usepackage[backend=bibtex, style=authoryear, natbib=true]{biblatex}
%\addbibresource{CCH.bib}

%%For Hypotheses and Subhypotheses
%\usepackage{ntheorem}
\newtheorem{hyp}{Hypothesis}
\makeatletter
\newcounter{subhyp}
\let\savedc@hyp\c@hyp
\newenvironment{subhyp}
 {%
  \setcounter{subhyp}{0}%
  \stepcounter{hyp}%
  \edef\saved@hyp{\thehyp}% Save the current value of hyp
  \let\c@hyp\c@subhyp     % Now hyp is subhyp
  \renewcommand{\thehyp}{\saved@hyp\alph{hyp}}%
 }
 {}
\newcommand{\normhyp}{%
  \let\c@hyp\savedc@hyp % revert to the old one
  \renewcommand\thehyp{\arabic{hyp}}%
}
\makeatother
%%%

\setcounter{MaxMatrixCols}{10}

\theoremstyle{plain}
\newtheorem{prop}{\protect\propositionname}
\theoremstyle{plain}
\newtheorem{thm}{\protect\theoremname}
\makeatother
\providecommand{\examplename}{Example}
\providecommand{\propositionname}{Proposition}
\providecommand{\theoremname}{Theorem}
\newcommand{\bi}{\begin{itemize}}
\newcommand{\ei}{\end{itemize}}
\newcommand{\bb}{\begin{block}}
\newcommand{\eb}{\end{block}}
\newcommand{\bmath}{\begin{eqnarray}}
\newcommand{\emath}{\end{eqnarray}}
\newcommand{\bmathnn}{\begin{eqnarray*}}
\newcommand{\emathnn}{\end{eqnarray*}}
\newtheorem{theorem}{{Theorem}}%[section]
\newtheorem{proposition}[theorem]{Proposition}
\newtheorem{lemma}[theorem]{Lemma}
\newtheorem{definition}[theorem]{Definition}
\newtheorem{prediction}{Prediction}
\newtheorem{open_question}{Open question}
\newtheorem{assumption}[theorem]{Assumption}
\newtheorem{observation}[theorem]{Observation}
\newtheorem{claim}[theorem]{{Claim}}
\newtheorem{example}[theorem]{{Example}}
\newtheorem{corollary}[theorem]{{Corollary}}
\newtheorem{remark}[theorem]{{Remark}}
\newtheorem{assumptions}[theorem]{{Assumptions}}
\newtheorem*{thm1star}{{Theorem $\mathbf{1^*}$}}
\newtheorem*{thm2star}{{Theorem $\mathbf{2^*}$}}
\newtheorem*{theorem*}{Theorem}
\newtheorem*{lemma*}{Lemma}


%For nice tables
\usepackage{booktabs}
\usepackage{array}
\usepackage{threeparttable}
\usepackage{tabulary}
\newcolumntype{M}[1]{{\centering\arraybackslash}m{#1}}

%Expectation opperator
\newcommand{\Expect}{{\rm I\kern-.3em E}}

%Significance commands
\newcommand*{\SuperScriptSameStyle}[1]{%
  \ensuremath{%
    \mathchoice
      {{}^{\displaystyle #1}}%
      {{}^{\textstyle #1}}%
      {{}^{\scriptstyle #1}}%
      {{}^{\scriptscriptstyle #1}}%
  }%
}

\newcommand*{\oneS}{\SuperScriptSameStyle{*}}
\newcommand*{\twoS}{\SuperScriptSameStyle{**}}
\newcommand*{\threeS}{\SuperScriptSameStyle{*{*}*}}
\newcommand{\vh}[1]{\textcolor{red}{(VH: #1)}}
\IfFileExists{upquote.sty}{\usepackage{upquote}}{}

\begin{document}


\title{Mobile Internet and Nation Building in Africa\\  \vspace{0.75cm} \underline{Preliminary and incomplete }}

\vspace{1cm}

\author{
\textbf{Illan Barriola} \\ CRED -- Paris II \\
\and \textbf{R\'{e}dha Chaba}\\ LEMMA -- Paris II \\
}

\date{ \today}

\maketitle
\section*{Data}
We use Afrobarometer surveys to leverage geocoded information on public opinion, information-gathering and demographic characteristics across African countries at the individual level.\footnote{All Afrobarometer surveys were conducted via-in-person, face-to-face interviews. The method did not change to computer-assisted interviews, thereby avoiding potential bias toward areas with mobile internet coverage.} We rely on rounds 5 to 8, covering the period from 2011 to 2021, to construct our sample of 98,235 individual respondents across 20 Sub-Saharan African countries.\footnote{Our sample includes: Benin, Burkina Faso, Botswana, Cameroon, Ivory Coast, 
Ghana, Guinea, Kenya, Liberia, Mali, Malawi, Mozambique, Namibia, Niger, Nigeria, Sierra Leone, Tanzania, Uganda, Zambia, Zimbabwe.}\footnote{We exclude islands, countries with multiple capitals, and Northern African countries. We also exclude Burundi, Gabon, Gambia, and Senegal due to missing data on mobile internet coverage.}
We measure political accountability by focusing on two key sets of variables from the Afrobarometer.
First, we measure trust in institutions through the following questions:
\textit{‘‘How much do you trust each of the following: the president, the parliament/national assembly?’’}.
Each question is measured on a scale from 0 ('Not at all') to 3 ('A lot'). We construct our main dependent variable \textit{Institutional Trust} by averaging the responses to these two questions.\footnote{In Table X we report the results of the replication using the variables analyzed individually.}
Second, we measure the willingness to sanction the ruling party through the following question:
\textit{‘‘If [presidential elections] were held tomorrow, which [party’s candidate] would you vote for?’’}.
We create a binary variable that equals 1 if the respondent indicates they would vote for any party other than the ruling party, and 0 if they would vote for the ruling party.\\

The geographic spread of our sample, ranging from urban centers to remote rural areas, allows us to effectively examine how the impact of mobile internet on political accountability varies with distance from political centers.
We leverage the precise geocoding of Afrobarometer survey enumeration areas to calculate the distance of each survey respondent from their country's capital city. Similar to Michalopoulos and Papaioannou (2014), to account for varying country sizes in our sample, we construct a relative distance measure as our main independent variable instead of using absolute distance from the capital city. For each respondent, we calculate the relative distance by dividing the respondent's distance from the capital by the maximum distance to the capital within the country.
This standardization allows for meaningful comparisons across countries of different sizes. The resulting values range from 0 (at the capital) to 1 (at the furthest point from the capital), providing an intuitive scale of remoteness that is comparable across all countries in our sample.\\

We combine Afrobarometer geocoded individual level data with detailed digital maps of mobile internet coverage provided by the Global System for Mobile Communications Association (GSMA). These maps compile coverage data from mobile network providers to measure coverage availability at a reliable signal strength with a 1x1-kilometer grid cell resolution.
To integrate mobile network coverage data with the Afrobarometer surveys, we calculate regional mobile internet coverage for each region and year. This process involves overlaying the mobile internet coverage map with population density data at the grid-cell level. We then compute a weighted average of mobile internet availability for each region, using population density as weights. These weights are normalized to sum to one, ensuring comparability across regions of different sizes and population distributions.

\section*{IV method}

Specifically, our main treatment of interest is the interaction between internet coverage at the regional level and distance from the capital city. This approach allows us to capture potential heterogeneity in the effect of internet news exposure across different geographic contexts.\\

Several econometric challenges necessitate an instrumental variable (IV) approach. Reverse causality is a concern, as trust in political institutions may influence internet news consumption patterns, creating a bidirectional relationship that complicates causal inference. Omitted variable bias poses a threat, as unobservable factors such as political interest, media literacy, or cultural attitudes may affect both news consumption and institutional trust, potentially biasing our estimates. Furthermore, measurement error in self-reported internet news consumption could lead to attenuation bias. These issues are compounded when considering the interaction term, as any endogeneity in the component variables carries over and potentially amplifies in the interaction.\\

To address these challenges, we propose using the interaction between mobile internet coverage and distance from the capital city as our instrument. This choice is motivated by several factors. Higher mobile internet coverage directly increases the potential for internet news consumption, ensuring relevance. The rollout of internet infrastructure is primarily driven by commercial and technical factors rather than political considerations, enhancing the likelihood of satisfying the exclusion restriction. Moreover, mobile internet coverage and its impact likely vary with distance from the capital city, providing a strong first-stage relationship with our treatment of interest.
The interaction term in our instrument captures how the availability of mobile internet coverage changes with distance from the capital city, which in turn affects the likelihood of internet news consumption across different regions. This approach allows us to estimate how the influence of internet news consumption on trust varies depending on geographic proximity to the center of political power, using only the variation in news consumption driven by exogenous factors related to internet infrastructure development.\\

While our instrument choice is theoretically justified, we acknowledge potential criticisms. In some contexts, governments may influence internet infrastructure deployment for political reasons, potentially violating the exclusion restriction. Additionally, uneven internet development could reflect broader patterns of regional inequality, which might independently affect trust in institutions. The interaction with distance from the capital could exacerbate these concerns if there are systematic differences in how mobile internet infrastructure is rolled out about proximity to the political center.\\

To address these concerns, we argue that economic and technical determinants are the primary drivers of the expansion of mobile internet coverage, rather than political incentives. Investment decisions made by telecom providers — often international firms – are made on the basis of their economic viability and expected returns. This emphasis on financial incentives diminishes the probability of politically motivated deployment strategies. Additionally, variations in mobile internet coverage across regions often arise due to technical and geographical limitations. Terrain, population density and the availability of supporting infrastructure all define the feasibility and cost of network deployment. These technical considerations often override any potential political motivation in infrastructure development decisions. We capture the nuanced interplay of these economic and technical factors across diverse geographic contexts by using an instrument that interacts mobile internet coverage and distance to the capital.\\


\end{document}